\chapter{Życie II}
\chapterAuthor{efem}

Smród. Pierwsze, co poczułem, kiedy wracały mi skrawki świadomości. Smród, który spotkać można było nad jeziorem w~wyjątkowo upalne dni. Jak gdyby zebrać wszystko, co porasta muliste nadbrzeże, wymieszać z błotem i~zostawić na kilka dni w~pełnym słońcu. W~normalnych okolicznościach każda część ciała chciałaby jak najszybciej się od tego uwolnić, zostawić to za sobą, jednak w~tej konkretnej chwili o~tym nie myślałem. Do smrodu człowiek szybko się przyzwyczaja. Nieraz zastanawiałem się, jak to jest pracować w~fabryce karmy dla psów, gdzie zapach w okolicy dwóch kilometrów jest nie do zniesienia. A jednak ktoś tam pracuje, znosi to -- przyzwyczaił się. Człowiek to istota posiadająca niesamowite zdolności adaptacji, a~węch akurat przystosowuje się bardzo szybko. 

Ciemność. Na początku, do końca nie wiedziałem, co powoduje tę ciemność -- czy były to środki farmakologiczne, czy to całe pomieszczenie jest odcięte od świata i~nie pozwala żadnemu sygnałowi z~zewnątrz tutaj dotrzeć. Ale gdzie było właściwie zewnątrz, a~gdzie wewnątrz? Gdzie byłem ja? Otoczenie było nieprzeniknione i jednorodne. Bez jakiegokolwiek wyłomu w~swej doskonałości. W~młodości interesowałem się kosmosem, gwiazdami i~wszystkim, co z~tym związane i~gdyby okoliczności były inne, próbowałbym się pewnie wczuć w~rolę kosmonauty, który po nieudanej wędrówce kosmicznej, utknął w przestrzeni międzyplanetarnej, a~otaczająca go ciemność, stara się nad nim zawładnąć, tocząc nierówną walkę z~jego umysłem i~emocjami. Tutaj sytuacja była podobna tylko w~jednym wymiarze: ta walka na pewno nie była równa. Otwierając drugie oko, poczułem, że coś chce mi w~tym przeszkodzić. Chciałem sięgnąć ręką w~okolice łuku brwiowego, żeby to zdjąć z~siebie i~odrzucić jak najdalej, ale… 

Ciało. Jakby nie moje. Na początku nie mogłem wykonać żadnego ruchu. Jakby jakaś siła w~tej ciemności miała inne plany i~pozwalała jedynie na otwarcie jednego oka, krępując resztę mojego ciała cuchnącą masą, na tyle szczelną, że po początkowym gwałtownym ruchu ręki, choćby o~centymetr, wracała ona zaraz w~to samo miejsce, jakby zassana przez podciśnienie. Po kilkunastominutowej walce o~jakiekolwiek oswobodzenie, która z~zewnątrz musiała wyglądać dość komicznie -- konwulsyjne ruchy ślepca, który stara się w~jakikolwiek z dostępnym sposobów zyskać choć trochę przestrzeni, choć trochę zrozumieć, co mu się przytrafia. 

Otoczenie. W~początkowych chwilach mojej walki miałem wrażenie, że niedostrzegalne ściany są coraz bliżej mnie, ciężko mi było złapać oddech, serce waliło niczym skrzydła kolibra, a~w~uszach słyszałem jedynie pisk. Pisk pochodzący, jak się okazało w~całości z~wnętrza mojej głowy, bo gdy sie uspokoiłem otoczenie i~jego odgłosy zaczęły rzeczywiście przypominać przestrzeń kosmiczną -- zero jakichkolwiek bodźców z~zewnątrz. Pomyślałem, że to na pewno sen i~już czekałem, kiedy będę mógł się obudzić i~wypić solidny kubek mocnej kawy. Jednak podczas oczekiwania na wybudzenie, które wcale nie nadchodziło, moje zmysły zaczęły się uspokajać, a~ja sam niejako zaakceptowałem ciemność i~wszystko, co z~tym związane. Serce biło wolniej, szum w~uszach ustawał. Zaczęło mi dokuczać zimno i~wtedy zdałem sobie sprawę, iż leżę zanurzony w dziwnej, śmierdzącej mazi. Leżę nagi. Będąc ułożonym na wznak, z~rękami po bokach torsu, poczułem na policzkach ciepło własnego oddechu. Dmuchnąłem mocniej w~górę -- powietrze się odbiło i~okoliło moją twarz niczym ciepła mgła. Zrozumiałem, że nade mną musi znajdować się swego rodzaju wieko zamykające cały pojemnik, który był na tę chwilę moim domem. A więc tym sposobem mnie tutaj włożyli. Tylko kto i~po co? 

Postanowiłem znowu spróbować się poruszyć, tym razem wolniej. Przypomniałem sobie zasadę cieczy nienewtonowskiej, gdzie przy dużym nacisku na nią, przyjmuje cechy ciał stałych, a~przy delikatnym ruchu zachowuje się jak wodnisty kisiel. Początkowo bez skutku, jednak po pewnym czasie poczułem, że moja ręka się przesuwa! Wędruje ku górze a~ja~z oczami wlepionymi w~miejsce, które powinno być moim ramieniem, oczekuję je ujrzeć. Nie dociera do mnie, że przecież nic nie widzę i~nic nie zobaczę. Ani ręki, ani nic innego. Pomimo, jak mi się wydało, pół godziny spędzonej w~egipskich ciemnościach, dalej nie widziałem zupełnie nic. Po jakimś czasie i~solidnej porcji walki ze śmierdzącą breją, udało mi się wyswobodzić jedną rękę. Czułem, jakbym odzyskał wolność, jakbym nie miał jej przez sto lat, a~teraz została mi przywrócona. Niesamowite uczucie wynagrodzenia za wysiłki. Uwolnioną rękę od razu wyciągnąłem ku górze i~rzeczywiście, na wysokości około czterdziestu centymetrów nade mną, znajdował się swego rodzaju właz czy wrota, które pchnięciem próbowałem pokonać. Szybko jednak okazało się, że jedną ręką to można, co najwyżej się podetrzeć, a~nie wyważyć solidne wieko. Skupiłem się zatem na wydostaniu drugiej ręki i~po kolejnych kilku minutach walki, miałem już obie kończyny na wierzchu. 

Walka. Po wydostaniu obu rąk i~uspokojeniu oddechu, chciałem jak najszybciej wydostać się z~tej kaźni. Wyciągnąłem ręce do góry, położyłem je płasko na wieku. Był to solidny kawał metalu, zimny i~chropowaty od rdzy. Kiedyś pewnie pomalowano go na jakiś kolor i~służył czemuś innemu, niż przetrzymywaniu ludzi w~zamknięciu. Dłonie oparte całą powierzchnią, łopatki silnie zaparte o~dno zbiornika i~rozpocząłem naciskanie od dołu na olbrzymi właz. Na moją niekorzyść działało nie tylko to, że smolista substancja uciskając na klatkę piersiową pozbawiała mnie oddechu, sprawiając, że każdy haust powietrza musiał być wywalczony, ale też to, że nie mogłem do końca wyprostować łokci, właściwie to w znacznym stopniu miałem je zgięte i~z~całych sił starałem się unieść wieko, choćby na milimetr. Parłem z~całych sił, zaciskając zęby, napinając do granic moich możliwości każdy mięsień, który mógł mi w~tym pomóc. Plecy i~brzuch musiały być twarde jak stal, ale to ręce wykonywały największą pracę. Próbowałem zwiększać siłę falami, za każdym razem wizualizowałem sobie, jak moje wysiłki zostają zwieńczone wydostaniem się z~niewoli. Po kilkunastu bezowocnych próbach zacząłem opadać z sił -- wieko ani drgnęło. Nie mogłem w~to uwierzyć, nie mogłem się z tym pogodzić. Jak to, ja? Miałem utknąć tutaj na zawsze? Oddychałem coraz ciężej, niewidoczne ściany znowu zbliżały się do mnie i~było to nie do zniesienia. Zacząłem krzyczeć i~walić pięściami w~wieko. Szarpałem się jak w ukropie, marnując resztki sił na bezładne i~nieskoordynowane próby ucieczki, jednak coś mi mówiło, że to na nic. Ostatnie tchnienie wiosny. Kiedy już miałem stracić całą nadzieję, poczułem, że w~pewnym momencie sarkofag, w~którym byłem uwięziony, przesunął się po podłodze. Specyficzny, bardzo ostry i chropowaty dźwięk metalu przesuwanego po surowym betonie, był pierwszym dźwiękiem, jaki usłyszałem od powrotu świadomości, który nie został wydany przez moje ciało. W~pierwszej chwili nie widziałem w~tym żadnej szansy, mózg osłabiony ogromem myśli, zmartwień i~różnych scenariuszy na to, co może się ze mną za chwilę wydarzyć potrzebował kilku chwil, aby dodać dwa do dwóch. Pomyślałem, że warto spróbować odpowiednio balansując ciałem, przechylić pojemnik ze mną i~resztą jego zawartości tak, aby znalazł się na boku, a~wtedy wieko, które możliwie, że jest jedynie nałożone na górę, bez żadnych zamków po prostu zsunie się z~pomocą grawitacji. Zacząłem kołysać moim okrętem w~taki sposób, jak kiedyś straszyłem dziewczyny na łódce -- przenosiłem balans z lewej strony na prawą i~z~powrotem. Początkowo bez wzruszenia, po chwili jednak pojemnik zaczął odpowiadać i~znalazłem z~nim wspólny rytm. Resztkami sił przerzucałem się od jednej do drugiej burty, coraz szybciej, z~coraz to większym impetem, aż w~pewnej chwili poczułem, że jeden bok odrywa się od ziemi. To jeszcze było za mało, ale dodało mi wiary w~siebie i~z~jeszcze większym zacięciem zmusiło do kontynuowania kołyski. Po chwili kontener skakał z~jednego boku na drugi, aby w~pewnym momencie stracić równowagę i~wylądować na boku. 

W~jednej chwili usłyszałem ogromny hałas, jakby walącej się ściany zbudowanej z~żelaznych cegieł, a~na twarzy i~reszcie ciała poczułem świeże powietrze. Musiało się stać to, na co miałem nadzieję i~niczym nieprzytwierdzone wieko zsunęło się z~przewracającego się pojemnika, dając dostęp do życiodajnego tlenu. Breja, w~której byłem zanurzony wylała się wraz ze mną, przynosząc ulgę moim nogom, które przez ostatnie kilka godzin leżały jak martwe, pod warstwą swego rodzaju mułu. Mogłem wreszcie głęboko oddychać, w~dodatku świeżym powietrzem. Leżałem na betonowej podłodze, na śmierdzącej i~kleistej pierzynie i~bezgłośnie śmiałem się sam do siebie, że udało mi się stąd wydostać. W~uszach słyszałem bicie serca, poranione ręce pulsowały tym samym rytmem, a~skóra paliła żywym ogniem. Odpocznę jeszcze trochę -- pomyślałem patrząc w~sufit. Moją sytuację bardzo szybko przyszło mi ocenić, kiedy usłyszałem coś na kształt rozmowy dobiegający jakby zza ściany. W~jednej chwili zerwałem się, przywracając mój tułów do pionu, z~nogami nadal spoczywającymi na ziemi, które wydawały się nienaturalnie lekkie i~gdyby przyszło mi w~tej chwili uciekać. Byłem jednak pewien, że dziesięć kroków byłoby wszystkim, co bym dał radę w~tej chwili pokonać. Zacząłem rozglądać się wokół, poszukując jakiegoś punktu zaczepienia. Nadal nie dało się dostrzec ani jednej rzeczy w~ciemności, nie widziałem swoich dłoni, które zbliżyłem na nie więcej, niż cal od twarzy. Dotykiem zbadałem pojemnik, w~którym spędziłem kilka ostatnich godzin. Okazało się to być swego rodzaju wanną o długości około trzech metrów i~szerokości dwóch, wyposażona u! podstawy w~komplet metalowych nóg. W~normalnych okolicznościach na pewno nie dałbym rady tego przewrócić, jednak jej stabilność została zachwiana przy pomocy nóg, na których opierała się cała konstrukcja. Twórca podnosząc środek ciężkości, nieświadomie pomógł mi w~ucieczce. 

Zanim do końca zbadałem moje otoczenie, już na czworaka, po raz drugi usłyszałem mocno stłumione głosy ludzi. Tak, to na pewno była rozmowa. Dało się wywnioskować, że jest to raczej wydawanie poleceń przez przełożonego, niż luźne pogaduchy przy meczu. Zamarłem w~bezruchu, po kilku sekundach dźwięki ustały, a~pokój zaczął nabierać kształtów i~kolorów. Odwróciłem głowę i~na drugim końcu pomieszczenia przy podłodze pojawił się wąski, święcący pasek. Drzwi -- pomyślałem. Z~jednej strony ucieszyłem się na myśl, że jest stąd wyjście i~chciałem krzyczeć o pomoc, jednak po chwili przyszło opamiętanie i~obawa, że ten, kto na nie odpowie wcale nie musi być moim przyjacielem. Co więcej, raczej na pewno nim nie będzie. 

W~tej chwili zacząłem odtwarzać przebieg wydarzeń sprzed uprowadzenia. Jednak poprzedni dzień nie różnił się zasadniczo niczym od szeregu dni go poprzedzających. Pobudka wcześnie rano, przebieżka z~Aliasem wzdłuż wybrzeża, krótsza niż zazwyczaj, bo przerwana nagłym deszczem. Gdyby to zależało tylko ode mnie, biegłbym dalej, ale Alias to straszna maruda, nie lubi deszczu, w ogóle nie lubi się moczyć. Kąpanie go jest straszną udręką, a~wzięło się to chyba stąd, że będąc szczeniakiem stąpając po wiszącym lodzie, wpadł do strumyka. Więcej było strachu niż szkód, ale awersja mu pozostała. Po śniadaniu standardowo praca -- oddanie projektu poszło gładko, a~na zwieńczenie dnia spotkanie z~Oliwią, która przyjechała do mnie późnym popołudniem, od razu zastrzegając, że nie zostanie na noc, bo rano musi coś załatwić w~swojej części miasta. Wypiliśmy wino słuchając muzyki i~rozmawialiśmy o~minionym dniu. Wspomniała coś o~nowej wystawie tandemu artystów Szlezwika i~Holsztyna, na której byliśmy w~zeszłym roku i~bardzo mi się podobała, więc zaproponowała, żebyśmy poszli w~moje urodziny, które już za tydzień, na co przystałem z~ochotą. Potem obejrzeliśmy odcinek jej ulubionego serialu i~odwiozłem ją na kolejkę. Wróciłem do domu, spacer z~Aliasem, wymiana smsów~z dziewczyną i~sen.

Oliwia nie zna mojej tajemnicy, nikt jej nie zna. Sądzę, że nie ma takiej potrzeby, a~na pewno jeszcze nie teraz. Od ostatniej śmierci z~rąk domorosłego łucznika, staram się uważać na siebie i~na moich bliskich ze zdwojoną siłą. Obecnie zbyteczne jest obarczanie jej wiedzą o~tym, że zdarzało mi się umrzeć, a~potem odrodzić w~nowym ciele, w~nowej sytuacji, do której musiałem się dostosować. Tym razem trafiłem na dwudziestosiedmioletniego Roberta, co przy jego obecnym wieku, daje osiem lat w~moim nowym ciele. Bardzo dobre ciało -- wzrost ponad 190~centymetrów, silna muskulatura -- okaz zdrowia. Postaram się tego nie zmarnować, chociaż to samo mówiłem o~poprzednich skórach. Stąd ten codzienny jogging, zdrowe odżywanie, kursy tańca, nauka chińskiego dla dobrej kondycji umysłowej. Robert nie ma dużej rodziny; rodzice mieszkają w~Kanadzie, a~jedyny brat robi karierę w~jednym ze Szwajcarskich banków. Kiedyś odwiedzałem moje poprzednie rodziny, moich starych znajomych, sprawdzałem, jak wygląda ich świat po moim odejściu, po odejściu jednego z~moich wcieleń. Reakcje, to zrozumiałe, były różne. Od depresji u~jednych, poprzez obojętność i~szybkie pogodzenie się z~tym, aż po uczucie ulgi u~niektórych. Z~czasem oswoiłem się z~tym, co zaowocowało nawykiem pogłębiania relacji z~wąską grupą ludzi w~moim otoczeniu. Staram się nie marnować czasu ani energii na nieznaczące znajomości czy beznamiętne uśmiechy w~pracy. Z~Oliwią jest mi dobrze, uważamy, że szczerość i~rozmowa są najważniejsze. Tak, szczerość -- brzmi to trochę ironicznie w~kontekście mojej historii, ale jeszcze nie teraz, jeszcze na to za wcześnie. Byłoby wielkim żalem stracić ją teraz, kiedy jest nam tak dobrze.

Kiedy oczy patrzą przez kilka godzin w~ciemność, najmniejszy promień światła rozbudza zmysł do tego stopnia, że człowiek czuje, jakby pozyskał nadludzkie możliwości. Widzenie zwykłych kształtów odbiera się jak zdolność przenikania wzrokiem przez ściany. To delikatne światło wydostające się spod drzwi na końcu pokoju, pozwoliło mi pozyskać nieco więcej informacji odnośnie tego, gdzie aktualnie się znajduję. Wanna, w~której leżałem stała na środku pokoju, a~po jego bokach z~obu stron ustawione były regały przemysłowe o~regulowanej wysokości półek. Na tych po lewej stronie panował kompletny nieporządek, nie mogłem rozpoznać, co się tam znajduje. Kształty były na tyle różne, że jednym razem wskazywały na narzędzia przemysłowe, a~innym na bezkształtnie rzucone w~kąt tkaniny czy futra. Za to regał po prawej stronie zaciekawił mnie do tego stopnia, że postawiłem spróbować zatrudnić do pomocy nogi, które pokryte zastygającą szarą mazią, aż prosiły się o~wykorzystanie. Ostrożnie, uważając, aby się nie przewrócić, wstałem i~podszedłem do regału. Wytężając wzrok i~korzystając z~moich dłoni, które w~ciemności zdały się kilkukrotnie bardziej czułe na dotyk, zacząłem badać jego zawartość. Na dolnej części stały jakby wysokie buty, obok coś na kształt protez ludzkich nóg. Trochę chropowate, pewnie od tej wszechobecnej wilgoci i~starości -- coś musiało się na nich osadzić. Piętro wyżej zajmowały różnego rodzaju kosmetyki, przynajmniej kształt butelek na to wskazywał. Nie mogłem rozpoznać zapachu, był podobny do zapachu plastiku, przeszywającej chemicznie sztuczności. Ostatnie piętro zawierało dość osobliwy zbiór figur, które wyglądały na modele używane do prezentacji peruk. Głowy stojące na regale, niektóre przystrojone sztucznymi włosami, niektóre jeszcze łyse. Dziwnie przedstawiały się te bez oczu, przypomniał mi się obrazek lalki z~wydłubanymi oczami, widziany gdzieś w dzieciństwie. Te z~oczami spoglądały bezmyślnie przed siebie, bez żadnego wyrazu, jakby uchwycone w~niebycie. 

Następnie podszedłem do drzwi. Chciałem im się bliżej przyjrzeć, spróbować znaleźć jakiś słaby punkt, może nawet je otworzyć. Była to solidna drewniana konstrukcja na ciężkich zawiasach. Klamka była okrągła, kiedyś musiała być bladozłota, teraz metal starł się i~widać było jedynie moje wykrzywione oblicze na sino-srebrnej sferze. Z~mojej strony nie było żadnego zamku, widocznie nie zaprojektowano ich do przetrzymywania więźniów i~były rozwiązaniem tymczasowym, przejściowym miejscem mojej kaźni. Pomyślałem, że zanim dokonam próby ich pokonania, dobrze by było wyposażyć się w~jakieś odzienie i~może coś do samoobrony po opuszczeniu pokoju.

Wróciłem do regału po lewej stronie w poszukiwaniu jakiegoś okrycia i~narzędzia, które pomogłyby mi w~sforsowaniu drzwi. Znalazłem coś, co przysposobiłem na swoje odzienie nawiązujące do ludów prymitywnych. Był to kawał materiału, z~którego wydarłem dużą, pokrytą smarem przepaskę. Przerzuciłem ją przez ramię, przełożyłem między nogami i~zawiązałem na wysokości brzucha. Nie było to wiele, ale lepsze to niż nic. Wśród różnego rodzaju narzędzi, z~których większość była połamana, jedyną pomocą mógł służyć ręczny sekator. Wziąłem do go ręki, leżał dobrze. Spojrzałem po sobie jeszcze raz: sine stopy, nogi pokryte ciemną zastygającą breją, dalej cuchnąca szmata ledwo zakrywająca tors, a~w~ręku narzędzie ogrodników. Twarz musiałem mieć czerwoną, niczym flagi na pochodzie pierwszomajowym, paliła mnie z~nerwów. Ciało jakby podzieliło się na dwa obozy: dolny, w~którym były nogi i~tors, które nadal odczuwały zimno i~niepewność, trochę drżały i~z~oporem przyjmowały rozkazy wydawane im przez głowę. Górny obóz, czyli właśnie głowa, pulsowała jak czerwona żarówka podczas alarmu bombowego w~schronie. Było mi gorąco i~zimno zarazem, specyficzny stan pomieszania lęku, oczekiwania i~euforii, który wyzwolił się na chwilę przed próbą sforsowania drzwi.

Podszedłem do drzwi, ostrożnie złapałem za klamkę i~starając się nie wydobyć z~niej żadnego dźwięku, przekręciłem ją powoli w~prawo. Drzwi drgnęły i~zaczęły się powoli otwierać. Czyżby to było aż tak proste? -- pomyślałem. W~pierwszej chwili sam sobie odpowiedziałem twierdząco, co dodało mi animuszu i~szarpnąłem drzwi mocniej, kiedy w~jednej chwili doniośle zadzwonił łańcuch, zawieszony na wysokości mojej głowy. Zamarłem. Za wszelką cenę chciałem uniknąć wydawania jakiegokolwiek dźwięku, ale na tym polu poległem. Łańcuch wydawał się niezbyt solidny, była to zapewne zwykła zasuwka antywłamaniowa, która pozwalała zajrzeć do środka pomieszczenia, jednak przełożenie ręki na zewnątrz było już wyzwaniem. W~pierwszej chwili starałem się przecisnąć dłoń na zewnątrz i~spróbować zdjąć blokadę ręcznie. Jednak konstrukcja przewidziała takie próby i~aby przesunąć blokadę zakotwiczoną na końcu łańcucha, trzeba było zmniejszyć kąt otwarcia drzwi, zapewne aż do ich całkowitego zamknięcia. Więc odrzuciłem ten pomysł i~przyszedł czas na wykorzystanie podręcznego sekatora. Chwyciłem za jedno z~ogniw i~z~całej siły zawiesiłem na nim swój ciężar. Ostrze narzędzia powoli wgryzało się w metal -- do ostatecznego przerwania jego ciągłości. Sekator został oznaczony dość głęboką szczerbą na obu jego ostrzach, ale nie było to ważne, liczyło się to, że blokada puściła i~byłem o jeden krok bliżej do wolności.

Wyszedłem na korytarz. Ostrożnie rozglądając się, starałem rozeznać w~sytuacji. Pomieszczenie było długości około dziesięciu~metrów i~szerokości dwóch, do tego nie było oświetlone. Na obu dłuższych ścianach umieszczono symetrycznie, naprzeciw siebie trzy pary drzwi podpisane u~góry numerami, z~których widziałem 202, 204 i 206. Na korytarzu panowała prawie całkowita ciemność. Jedynie za sprawą tego, że ostatnie drzwi na mojej ścianie musiały być przeszklone, bo biła od nich delikatna łuna oznaczająca czyjąś obecność, dało sie rozpoznać zarysy drzwi i~podłogi. Jeden koniec korytarza był zaślepiony, natomiast drugi, przy którym biło owe źródło światła, był zakończony dużym wejściem sygnowanym tabliczką EXIT. Zobaczyłem w~tym moją szansę na opuszczenie tego koszmarnego miejsca. Dłoń trzymająca sekator bezwiednie zacisnęła się i~powoli ruszyłem w~stronę lekko migoczącej żarówki. W~miarę zbliżania się, coraz wyraźniej słyszałem głosy dobiegające z~przeszklonych drzwi. Coś, co początkowo było zwykłym bełkotem, teraz przybierał coraz bardziej namacalny kształt rozmowy dwóch osób, jakby konsultacji. Będąc tuż przy samych drzwiach dało się słyszeć taka oto wymianę zdań:

---~Nie tak! Musimy z~tym uważać, powoli --~powiedział mężczyzna o~bardzo niskim głosie. Sprawiał wrażenie mającego przewagę nad tym drugim. 

---~Teraz jesteś taki mądry, ale kiedy mówiłem, żeby przyjechać tu ze swoimi narzędziami, to odmówiłeś --~odburknął mu ktoś, kto wydał mi się osobą wykonującą polecenia pierwszego. 

---~Nie gadaj tyle. Jak chcesz to skończyć? 

---~Myślę, że rozetniemy całość do końca, skoro i~tak cała zawartość już wypłynęła. Tutaj można dać wycięcie o~tak, tamto zaszyjemy i~wrzucimy do formy. 

---~Dobra niech będzie i~tak jesteśmy w~niedoczasie. Potem musimy zając się nowym, zaglądałeś do niego? --~zaordynował szef.

---~Jak mu tam, Robert? Nie, od rana nie miałem czasu, sam wiesz ile mamy roboty, a~czas nas goni. 

---~Racja, skończymy to i~podejdę do niego… 


Zrozumiałem, że mówią o~mnie, nie miałem, więc dużo czasu. Szyba była mleczna i~dziwnie pofałdowana, jakby pod wpływem znacznego ciepła, zaczęła się topić w~nieregularny sposób. Nie chciałem ryzykować, że ktoś dostrzeże moją sylwetkę przemykającą za drzwiami. Położyłem się na ziemi i~powolnymi ruchami czołgałem się w~kierunku wyjścia. Każdy ruch wykonywałem powoli i~starannie, uważając, aby odgłosy szurania po szorstkim betonie nie stały się na tyle głośne, by wzbudzić ich uwagę. Kiedy byłem u~celu, spojrzałem za siebie: na ziemi zostawiłem szaro-brunatny ślad z substancji, w~której się obudziłem. Dopiero teraz, w~słabym świetle rozpoznałem w~tym coś na kształt gliny. Bardzo gęstej~i jednorodnej, możliwe, że z~jakimiś dodatkami. Powoli rozchyliłem duże drzwi i~przerzuciłem swoje ciało na drugą stronę progu. Przede sobą miałem klatkę schodową. Schody prowadziły zarówno w~górę jak i~w~dół. Z~braku jakiegokolwiek punktu odniesienia, musiałem w~ciemno wybrać drogę ucieczki. Przypomniały mi się numery nad drzwiami, które mogły sugerować, że znajduję się na drugim poziomie. Ratunek zatem musiały przynieść schody w~dół.

Klatka schodowa pozbawiona była okien, tylko powieszone na półpiętrach nagie żarówki dawały trochę światła. Za oparcie dla rąk służyła bardzo zniszczona, metalowa rama przykręcona do schodzących w~dół wilgotnych ścian. Schodziłem dość szybko, bez butów byłem niemal bezszelestny, nie musiałem się więc martwić o~ten aspekt ucieczki. Jednak wszystko ma swoją cenę i~kiedy już widziałem główne drzwi wyjściowe, poczułem silne ukłucie w~lewej stopie. W~półświetle podniosłem nogę i~odwróciłem ją podeszwą do góry. Wystawał z~niej zardzewiały, zakrzywiony gwóźdź, a~na posadzkę pode mną zaczęła kapać gęsta, ciemna krew. Ostrożnie złapałem za jego koniec i~powoli zacząłem wyciągać trzon gwoździa z~mojej stopy. Kiedy już cały był na zewnątrz, krew popłynęła obficiej, a~jej ciepło poczułem na drugiej nodze, na której opierałem od kilku minut swój ciężar. Korzystając z~sekatora i~ciucha, który miałem na sobie, odciąłem pasek i~zawiązałem na stopie, aby zatamować upływ krwi i~chociaż prowizorycznie zabezpieczyć ranę. Wyprostowałem prawą nogę i~starałem się na niej stanąć, jednak ból był zbyt silny. Do drzwi doszedłem kulejąc i~jednym pchnięciem, otworzyłem je. Na twarzy poczułem wilgoć i~zapach deszczu. Na podwórzu już ciemniało, a~ciemne chmury przesłaniały i~tak ledwo już widoczne niebo. Musiał się zbliżać wieczór, a~mijający dzień należał do wyjątkowo deszczowych i~zimnych -- wnioskowałem po ilości błota na dziedzińcu przede mną.

Plac okrążał z~trzech stron budynek, z~którego właśnie wyszedłem. Był to prostokąt, a~odstępy między ścianami wynosiły około pięćdziesiąt~metrów. Długości musiał mieć przynajmniej drugie tyle. Wyjście mieściło się akurat vis-a-vis otwartej części dziedzińca. Miałem dobre przeczucie odnośnie ucieczki. Zostało mi już tak niewiele, aby dopiąć swego: za dziedzińcem, w~oddali widać było drogę i~jeżdżące na światłach samochody. Ostatni wysiłek, aby dostać się do drogi, złapać stopa i~wydostać się stąd na zawsze. Zebrałem w~sobie wszystkie siły, jakie mi pozostały i~zacząłem biec najkrótszą droga do wolności -- na wprost przez cały dziedziniec. Pierwsze metry, kiedy nogi zaczęły grząźć w~zimnym błocie, szybko zweryfikowały moje plany o~szybkiej ucieczce zmuszając, aby każdy krok okupiony był jeszcze większym wysiłkiem. Po przebiegnięciu jednej trzeciej drogi, usłyszałem za mną dziwny dźwięk, jakby kłus konia, serce podeszło mi do gardła -- odwróciłem głowę. Dwa psy rzuciły się za mną w~pogoń, wściekle szczerząc zęby. Wszystko, albo nic -- pomyślałem i~wykorzystując ostatnie skrawki energii biegłem, co sił w~nogach. Zapomniałem o~bólu spowodowanym wcześniejszym zranieniem. Kiedy byłem może dwadzieścia metrów od końca podwórza, zauważyłem, że wcale nie był on otwarty na resztę świata. Był ogrodzony na całej szerokości siatką, która z~daleka zlewała się z~całym mrocznym otoczeniem i~dopiero teraz ukazał mi się pełen obraz tego, co mnie czekało -– musiałem sforsować metalowe ogrodzenie. W~tej chwili poczułem rwący ból w~prawej nodze. To jeden z psów zacisnął zęby na moim goleniu. Szarpnęło mną nagle i padłem jak długi na ziemię. Przypomniałem sobie, że nadal mam ostry sekator w~ręku i~zacząłem nim na oślep ciąć powietrze u~stóp. Po kilku próbach trafiłem bestię w oko -- zaczęła piszczeć i~wyć z~bólu, ale w~tym momencie dopadła mnie druga. Hałas generowany przez naszą trójkę, musiał wywabić moich oprawców, gdyż w~pewnej chwili usłyszałem okrzyk i~strzał. Psy zamilkły. Jeden musiał właśnie dożyć swych dni, a~drugi posłusznie wykonywał polecenie swego pana. Na tle ciemnego i~ciężkiego nieba, zobaczyłem podchodzącą do mnie zakapturzona postać. Następne, co poczułem to uderzenie w~głowę jakimś długim narzędziem i~przeszywający pisk w~uszach.

Obudziłem się w~moim pokoju. Nie, nie w~domu. W~pokoju z~wanną. Tym razem musiała zostać przysposobiona jako stół, odwrócona do góry nogami, a~na nich położono te ogromne i~ciężkie wieko, które wcześniej blokowało mi drogę ucieczki. Spocząłem na wspomnianym włazie i~przypięto mnie pasami, takimi samymi, jakimi przypina się chorych na schizofrenię w~szpitalach psychiatrycznych. Z~góry, centralnie nad moją głową została opuszczona przenośna lampa, świecąca tak silnym światłem, że łzy spływały mi po policzkach, a~jedynym, co widziałem była oślepiająca jasność. Po chwili usłyszałem kroki -- do pokoju weszły dwie osoby, to było tamtych dwóch mężczyzn. W~oszołomieniu słyszałem skrawki ich rozmowy: 

---~Prawie mu się udało ---~powiedział ważniejszy.

---~No, prawie. Tylko psa szkoda, co im powiemy?

---~Pierdolić co im powiemy! Skończmy cośmy zaczęli i~wracajmy przygotować lokal. Dostaną swoją działkę.

---~Masz rację. To jak, robimy zgodnie z~planem? 

---~Tak, ma poranione nogi, ale i~tak nie mieliśmy w~planie ich wykorzystać. Najważniejsze są głowa i~palce, a~te wyglądają w~porządku.

---~To zaczynam ---~wymamrotał pomocnik.

Ze strachu wytrzeszczyłem oczy najmocniej, jak było to możliwe, chciałem zobaczyć, co na mnie szykują, ale im bardziej wytężałem wzrok, tym gęstsze płynęły mi łzy, a światło raziło mocniej. Zobaczyłem krępą postać, podchodząca do mego ramienia: 

---~No już ---~usłyszałem…

Zobaczyłem tylko błysk ostrza, które zmierzało z~góry, by spocząć tuż pod moim podbródkiem. Oczy wywinęły mi się na drugą stronę i~poczułem ulgę. Ciemność zaczęła zalewać wszystko dookoła, a~mi było wszystko jedno. 

\paraSep

Znowu to spadanie i~ból. Z~biegiem czasu nauczyłem się do tego przyzwyczajać. Ciekawe kim będę tym razem -- pomyślałem. Procedura przechodzenia zawsze jest nieprzyjemna -- najpierw utrata świadomości, urwany film, jako skutek zakończenia poprzedniego życia. Z~doświadczenia wiem, że brak przytomności trwa najwyżej kilka dni, po których zaczynam odczuwać pęd powietrza -- na początku delikatne muśnięcia, ale już po chwili drastycznie wzbiera na sile -- jakbym został strącony ze snu, który odbywałem na szczycie wieżowca. Kiedy czuję, że zbliżam się do ziemi, przychodzi silny ból głowy, połączony z~nieopisanym szumem w uszach, podczas którego muszą być ładowane do mojego mózgu informacje, wspomnienia i~uczucia mojego nowego gospodarza. Po odzyskaniu przytomności trochę trwa, zanim całą nową wiedzę w~mojej głowie poukładam i~przyporządkuje do prawdziwego świata przede mną. Idealnie jest, kiedy dotknie mnie to w~domu, gdzie margines na poznanie otoczenie jest większy. Tym razem, jednak tak się nie stało. 

---~Artur, patrz tam! ---~usłyszałem, i~w jednej chwili poczułem szarpnięcie kobiecej ręki.
 
Rozejrzałem się dookoła i~szybko uświadomiłem sobie, że jesteśmy razem z~moją nową kobietą w~jakiejś galerii. Właśnie trwała, sądząc po tłumach, oblegana wystawa, a~ona ciągnęła mnie do osobliwej gabloty. W~ręku miałem ulotkę o~tytule „Nowa wystawa duetu Szlezwik \& Holsztyn”. Podeszliśmy wspólnie do gabloty -- stała tam jakby sztuczna głowa, jak z~muzeum figur woskowych, która ogolona była na łyso, patrząc beznamiętnie na wprost z~palcami przyklejonymi do łysej glacy, tworzącymi swego rodzaju irokeza. Poczułem się słabo, uderzenie krwi do głowy zwaliło mnie z~nóg. 

Przestraszona Anna padła ze mną na ziemię dopytując co się stało, ale jedynym, co udało mi się z~siebie wydusić zanim straciłem przytomność było: 

---~Jestem Robert, a~to jest moja głowa!
