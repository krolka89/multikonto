\chapter{Życie XI}
\chapterAuthor{M4lutki}

Znowu ciemność, znowu spadam. Ciekawe co teraz? Błagam, jakiś sławny aktor filmowy, lub piosenkarz, może być nawet jakiś celebryta. Chciałbym w~końcu zaznać hulaszczego życia, a~nie jak dotąd, trafiać na nosicieli z~rozwodami, problemami finansowymi, albo menstruacją jak wtedy gdy byłem nastolatką. Niestety, już po pierwszej chwili wiedziałem, że moje nowe życie nie będzie sielanką. Zaczęło się z~grubej rury. Ciemność, jakieś krzyki, głośne eksplozje coś gorącego,  zapewne metalowego, ciągle uderzającego w~moją szyję. Totalny chaos, kurwa mać -- moja świadomość ugrzęzła w~kimś, kto bez wątpienia jest w~tym momencie w~gównianej sytuacji.  

---~Tony! Tony! Wstawaj, obudź się! Kurwa, Tony błagam, daj jakiś znak! ---~brzmiał z~boku czyjś przerażony głos. 

Ktoś mnie woła -- przynajmniej nie jestem sam… To już coś. Po doświadczeniu z~sarkofagiem wypełnionym jakimś gównem, nauczyłem się, że wszystko jest lepsze od tamtego poczucia samotności i~zwątpienia. Próbowałem otworzyć oczy, ale powieki wydawały się jakby były zrobione z~ołowiu. Podniosłem głowę. „Dalej, dalej, oczy Gadżeta” -- próbowałem zmusić swoje powieki do rozwarcia i~wpuszczenia chociaż odrobiny światła do oka, bym mógł dokładnie zorientować się w~sytuacji, w~której ugrzęzłem tym razem. Z~niewiadomego mi powodu obraz, który ujrzałem, był cały czerwony. Sięgając ręką do czoła poczułem kleistą maź. Krew? To by tłumaczyło, dlaczego widzę w~takim kolorze. Jest gorzej niż myślałem, nie dość, że jestem w~tak posranej -- bo tak mi się wydawało -- sytuacji, to jeszcze uszkodzony na starcie. Zajebiście. 

---~Tony! Kurwa jego mać, Anthony Sulivan, wstawaj i~weź się w~garść, jesteś nam zajebiście potrzebny! ---~znowu ktoś do mnie krzyczał.

Zanim ostrość widzenia powróciła, zdążyłem rozeznać się w~życiorysie mojego nowego nosiciela. Był nim Anthony Sulivan, obywatel USA i~tegoż państwa przedstawiciel na froncie. Żołnierz. Kawaler, syn zasłużonego pułkownika US Army, który poszedł w~ślady ojca i~postanowił spróbować swoich sił w~szeregach armii. Raczej to jeden z~tych typów twardzieli, nie bojący się wyzwań i~łaknący adrenaliny. W~końcu udało mi się dostrzec, gdzie jestem. Była to wojskowa baza USA w~prowincji Oruzgan w~Afganistanie. W~pozycji półsiedzącej leżałem oparty o~metalowy kosz wypełniony kamieniami, służący za dość skuteczne ogrodzenie i~ochronę, oddzielającą bazę od reszty, dzikiej, prowincji. W~około mnie leżały szczątki budki wartowniczej, a~obok mnie stał Dan, mój przyjaciel i~brat w~armii -- to on ciągle nawoływał moje imię.

---~Tony, bierz swój karabin i~pomóż!

---~Co się stało?

---~Zaatakowali nas! Ciapate skurwiele zaatakowały naszą bazę! Bierz karabin i~pokaż im kto tu rządzi!

Teraz wszystko stało się jasne. Był 7 sierpnia 2007 roku, a~ja tkwiłem w~wojskowej bazie w~najmniej gościnnym państwie na świecie, do tego podczas szturmu nieznanej nam liczby talibańskich bojowników. Z~tego, co zdołałem sobie przypomnieć, pocisk RPG trafił w~naszą budkę wartowniczą, ale jakimiś cudem udało nam się z~Danem uciec przed śmiertelną falą uderzeniową i~odłamkami. Musiałem wtedy stracić przytomność, bo gdy się ocknąłem faza ofensywy była już w~dość zaawansowanym stadium, a~nasze baza broniła się całych sił. Trzeba im to przyznać, mają wyczucie czasu -- akurat podczas mojej warty zachciało im się nas szturmować. Kurwa, trzeba było zostać w~domu.

Gorące coś, co ciągle uderzało w~moją szyję, nie było powietrzem, jak z~początku sądziłem. Były nimi łuski wylatujące z~karabinu Dana, który posiał już -- sądząc po ich ilości na ziemi -- dość pokaźną ilość ołowiu w~stronę wroga. Teraz czas na mnie! -- pomyślałem. Szczęśliwym trafem, moja broń leżała kawałek ode mnie. Chwyciłem „miotełkę” w~moje łapy. Był to karabin Mk14 EBR, używany przez jednostki specjalne i~piechotę US Army. Zasilany nabojem 7,72x51mm NATO, który mogłem celnie posłać na odległość mocno przekraczającą zasięg zwykłego żołnierza, za sprawą celownika teleskopowego Leupold Mark 4.  To był kawał świetnego karabinu, a~pieszczotliwy przydomek „miotełka” nadałem jej ze względu na to, że miała służyć w~jednym celu -- do sprzątania ciapatych. Wiecie, taki żołnierski humor, który w~tym momencie zupełnie się mi nie udzielał. Powolutku wychyliłem głowę ponad metalowy kosz. 

---~O~kurwa, Dan, nie jest dobrze ---~krzyknąłem.

---~No nie pierdol? Wczoraj się urodziłeś? To jest wojna! Weźmiesz się do roboty czy dalej będziesz tak stał, jak widły w~gnoju?

---~Rozkaz, panie kapitanie! ---~z~lekkim uśmiechem odpowiedziałem.

Spostrzegłem, że z~oddali zbliża się do nas ciężarówka. Talibowie czasami używali dużych samochodów dostawczych, wypełnionych materiałami wybuchowymi, by zrobić wyrwę w~ogrodzeniu, dzięki czemu mogli zdobyć możliwość wejścia do obiektu, który atakują i~to w~miejscu w~którym sobie teko zażyczą. Drugim wariantem mogło być, to, że to wcale nie ciężarówka-taran tylko jeżdżąca bomba, która mogłaby zrównać z~ziemią pokaźną część bazy. Tak czy siak, mieliśmy przesrane. Akurat południowo-wschodnia część bazy była idealnym miejscem na tego typu zagrywkę, a~my z~Danem i~resztą chłopaków, właśnie tej części mieliśmy bronić. To była moja szansa. Szansa na odegranie się za rozjebanie mojej zajebistej wieżyczki wartowniczej, w~której przesiadywałem całymi dnami przyglądając się tej zasranej pustyni… A~w~szafce miałem chipsy i~batony, które przegryzałem w~wolnych chwilach. 

---~Dan, ciężarówka, strzelaj w~ciężarówkę! Ja spróbuję zlikwidować kierowcę! – krzyknąłem.

Dan, wymieniając magazynek, tylko skinął głową i~zaczął ostrzeliwać pędzący w~naszą stronę pojazd. Ja tymczasem powolutku wychyliłem lufę „mioły” poza obręb kosza. Przystawiłem kolbę do dołka strzeleckiego i~spróbowałem wycelować. Na szkoleniu zawsze powtarzali, że nawet bicie serca wpływa na osiągi strzelca podczas oddawania strzału, dlatego należy się najpierw uspokoić a~dopiero później oddać strzał. Łatwo im to, kurwa, mówić, gdy w~prawdziwym starciu koło Ciebie przemykają pociski, które tylko czyhają by zatopić się w~twojej dupie, a~bazę pustoszą bomby moździerzowe. „Ok, Tony, uspokój się, wyceluj w~kierowcę i~ściśnij spust” -- powtarzałem. Tak też zrobiłem. W~obrębie mono okularu pojawiła się sylwetka kierowcy, który miał być moją pierwszą ofiarą tym wcieleniu, a~jedną z~wielu, którym życie odebrałem w~innych wcieleniach.

Nie miałem skrupułów. Ściągnąłem spust gdy tylko krzyż celowniczy zatopił się w~klatce piersiowej człowieka z~ciężarówki. Trafiony! Jeden mniej. Niestety ciężarówka nadal z~potworną prędkością pędziła w~naszą stronę. Biorąc pod uwagę to, że pewnie jedzie na maksymalnym gazie, wiec pokonuje jakieś 25-35 metrów na sekundę, to od kolizji z~naszą bazą dzieliły nas sekundy. Decyzja zapadła błyskawicznie.

---~Tony, to się nie zatrzyma! Pędzi prosto na nas, stary musimy stąd spierdalać inaczej wyparujemy! ---~oznajmił Dan.

---~Skurwiel czyta mi w~myślach ---~pomyślałem i~zacząłem krzyczeć: W~nogi!!! 

Uwierzcie mi. W~tamtym momencie nawet Usain Bolt był przy mnie pierdołą, a~struś pędziwiatr mógłby mi czyścić buty. Razem z~Danem i~resztą chłopaków znajdujących się w~pobliżu, biegliśmy ile sił w~nogach, by znaleźć się za jakąkolwiek osłoną. Część zdążyła dobiec do jednego ze schronów, a~część tak jak ja i~Dan, schowała się za uszkodzonymi MRAPami, które mechanicy parkowali nieopodal naszego punktu obserwacyjnego. Mimo, że nie jestem osobą religijną, w~tamtej chwili błagałem, by -- jeśli jakikolwiek bóg istnieje -- sprawił, że osłona z~którą się ukryliśmy wytrzymała. Nastąpił wybuch.
