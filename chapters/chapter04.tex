\chapter{Życie III, IV, V}
\chapterAuthor{Kozajsza}

---~You're going to be just fine Mr. Finchley ---~lekarz spojrzał beznamiętnie na kartę z~wynikami rezonansu magnetycznego.

---~Słucham?

---~Wszystko będzie w~porządku. To niewielkie uderzenie, a~omdlenie było prawdopodobnie spowodowane przemęczeniem. Proszę się tak nie forsować, żona mówiła, że ostatnio sporo pan pracuje.

No tak. Znam przecież angielski, byłem kiedyś anglikiem, teraz jestem Amerykaninem. Czasami opanowuje się ciało w~najmniej spodziewanym momencie. Mój osobisty numer jeden to bitwa pod Waterloo w~trakcie odwrotu. Żyłem okrągłe pięć minut, bo przecież gwardia umiera, ale nie poddaje się.

---~Dziękuje doktorze. Mogę już iść?

---~Oczywiście, lecz mam do pana jeszcze jedno pytanie. Mogę?

---~Słucham?

---~Zauważyłem lekką zmianę w~sposobie pana wypowiedzi, jest coś innego w~pana akcencie. Mój znajomy badał kiedyś pacjentów po omdleniach i~udarach. Czasem mają potem inny akcent, czasem nawet przyznają, że rozumieją języki, których wcześniej nie znali. Czy mógłby się pan z~nim spotkać? Na pewno bardzo by się ucieszył.

---~Przepraszam doktorze, ale nie sądzę, by zmienił mi się akcent ---~ewakuuję się powoli w~kierunku drzwi. ---~Upadając przyciąłem sobie język, myślę, że stąd lekka zmiana, wciąż czasami seplenię. Do widzenia doktorze.

\paraSep

Znam takich „badaczy”. Od czasu rozwoju medycyny w~XIX wieku niektórzy upodobali sobie wcielanie się w~takich właśnie ludzi. Urządzają sobie polowania, a~jeśli tylko wyłapią kogoś, kto zaszedł im za skórę -- kończy się to na wystawie w~galerii. Albo w~jakikolwiek inny spektakularny sposób, najlepiej taki, żeby w~następnym życiu ciężko było o~tym zapomnieć.

Wsiadam do samochodu, włączam radio i~ruszam do domu. Kwiaty dla Anny same się nie kupią, muszę więc zajechać do sklepu. Ciężko przeżyła mój wybryk w~galerii, dziwnie się zachowuje, boje się, że coś podejrzewa. Nie mogę się wydać. Nie podoba mi się to życie, ale muszę z~niego wydobyć jak najwięcej. Jako dyrektor finansowy firmy, która właśnie umoczyła dwa miliardy dolarów w~rozwijanie bezużytecznych technologii, jestem cały czas na świeczniku. Jeden głupi ruch i~wypadam z~gry, jedno głupie słowo i~resztę swojej wspaniałej egzystencji spędzę oglądając świat w~kratę. Przynajmniej miałbym czas, by spisać swoje przeżycia, chociaż wątpię bym był w~stanie przeżyć w~więzieniu. Podanie mojej twarzy do wiadomości we wszystkich mediach raczej zwróciłoby na mnie ich uwagę. Mam ochotę zabrać pieniądze z~banku i~spieprzyć w~najdalsze rejony świata, by zyskać choć trochę cennego czasu. Tego czasu, który jak tłumaczył mi T -- nie jest linią a~powierzchnią, po której swobodnie podróżujemy. Jest jednak rok 2047 i~raczej istnieją marginalne szanse na to, że w~ogóle znajdę skrawek tej planety, którego nie obejmuje wzrok kamer i~czujników, a~który nie jest jednocześnie strefą radiacyjną, czy jak to się kiedyś mówiło -- bliskim wschodem.

„Już ponad 80 procent bogactwa planety jest w~rękach zaledwie stu dwudziestu rodzin. Rozwarstwienie społeczne nasila się. Kolejne protesty wybuchają w~Chinach, zwłaszcza na terenach najbiedniejszych, dotkniętych ostatnimi wojnami z~Rosją i~Indiami. Jak podaje agencja Reuters, na ulicach może być nawet trzy miliony demonstrantów. Rząd Chin nie wyklucza użycia wojska”. Zawsze tak jest. Maluczcy się buntują, wierząc, że mogą coś wywalczyć. Wolność, niezależność, sprawiedliwość. Nie rozumieją, że to co zostało im dane, może zostać odebrane w~każdym momencie. To są wartości, które trzeba wyznawać, które trzeba wziąć i~bronić ich przez całe życie. I~nawet wtedy nie masz pewności. Nikt nie ma pewności tego co będzie. Nawet ja. Nawet ja kiedyś zniknę bezpowrotnie, a~moja dusza rozpryśnie się na tysiąc kawałków. Widziałem jak to się dzieje. Nie potrafiłem temu zapobiec. Zawiodłem.

Zjeżdżam na stację benzynową i~idę do łazienki, by zwymiotować. Nowe ciało próbuje odrzucić pasożyta, którym jest moja dusza. Wyrzygać coś, czego fizycznie nie ma. Trochę to potrwa, zanim się przyzwyczai. Patrzę w~lustro i~widzę twarz faceta po czterdziestce, szczęśliwego, ale jednocześnie zmęczonego ciągłą walką, którą prowadzi od dwudziestu lat. Walką ze sobą i~swoimi słabościami, walką z~otoczeniem, walką o~lepszy byt dla siebie i~bliskich. Który umrze już niebawem i~którego majątkiem podzielą się bliscy, szybciej niż jego ciało osiągnie temperaturę pokojową. Opłukuję twarz wodą i~zamykam oczy.

\paraSep

Ma twarz odbija się w~gładkiej tafli jeziorka, na zapleczu pałacu shoguna. Stoję na mostku, przyglądając się płynącym po jego powierzchni złotym liściom. Jesień przyszła szybko tego roku.

---~Jesień przyszła szybko tego roku ---~słyszę zza swoich pleców. Bladoniebieskie oczy Takiyamy -- doradcy shoguna, przecinają mnie spojrzeniem niczym wakizashi. Kłaniam się, choć mam ochotę rozciąć jego twarz na pół.

---~Shogun odmawia audiencji. Uważa, że sprawa jest przesądzona. Nie wpuścimy tu barbarzyńców. Twój trud jest równie bezużyteczny, co niepokojący. Nie chcesz chyba, by twój ród został uznany za kolaborujący z~cudzoziemcami i~innowiercami?

---~Jak śmiesz ---~wyduszam z~siebie. ---~To dla dobra moich ludzi. Czy nie widzisz dalej, niż te kilka lat w~przód? Chcesz bogactw dla siebie, wypędzicie cudzoziemców, a~za sto lat wrócą tu, kiedy kraj będzie głodował i~zdobędą go bez najmniejszych problemów. Jeśli ktoś działa na szkodę kraju to tylko ty!

---~Ta rozmowa jest skończona, Itagaki.

Patrzę na niego ze wściekłością jeszcze przez chwilę i~wychodzę. Schodzę po murowanych schodach pałacu tak wściekły, że prawie zapominam o~swoich mieczach. Odbieram katanę i~wakizashi od chłopca już za bramą i~udaję się w~kierunku parku. Muszę odetchnąć. Zastanawiam się czy szogun rozumie konsekwencje swojej decyzji. Mawia się, że w~krainie ślepców jednooki jest królem. Jesteśmy na etapie, gdy jednoocy chcą wyrwać mi oczy, bym stał się ślepcem. Przejmując ciało, często przejmuje się motywacje swojego „gospodarza”. Ten zaś jak nikt dba o~swój kraj i~swoich ludzi. Po ostatnich buntach chłopskich powinienem cieszyć się, że jeszcze mam głowę, a~jednak nadal chodzę do shoguna i~błagam go o~wysłuchanie. Jestem głupcem, z~kilkusetletnim doświadczeniem w~temacie. Żałosne.

Stoi na drugim końcu drogi, pomiędzy drzewami, pośród spadających na ziemię żółtych liści. Wiedziałem, że po mnie przyjdą. Takiyama się o~to postarał. Nie mam nic do stracenia. Gdy podchodzę bliżej, widzę, że to cudzoziemiec. Ubrany w~brązowe kimono, dzierżący miecze niczym samuraj, ale cudzoziemiec, prawdopodobnie holender, wnioskując po blond włosach zaczesanych do tyłu, w~stylu eleganckiego przedsiębiorcy.

---~Ik ben blij u degene die reist tussen tijd.

---~A ty jesteś?

---~Przyjacielem, wrogiem, ojcem, bratem, katem, wybawcą i~oprawcą.

---~Przysłał cię Takiyama?

---~Takiyama? ---~zaśmiał się grubym głosem. ---~Od kiedy pionki wydają rozkazy królom?

---~Kim więc jesteś?

---~Vous connaissez la réponse.

---~Przestań popisywać się znajomością języków. Nie imponuje mi to.

---~Tak myślałem. Ciężko zaimponować spartaninowi. Nieprawdaż Demastrosie?

---~Tym razem Itagaki.

---~Ten raz, tamten raz, cóż za różnica? Ten sam wilk w~wielu skórach nadal jest wilkiem, prawda? Demastros zaś miał bronić swej siostry. Cóż za paradoks. Wilk broniący owcy. Nic dziwnego, że nie podołał zadaniu.

---~Mojej siostry nie dało się obronić. Dobrze o~tym wiesz.

---~Ciebie też nikt nie uratuje. Nigdy. Będę zawsze. Będę wszędzie. By obrócić twoje szczęście w~popiół. Myślisz, że jesteś jedyny w~swoim rodzaju? Myślisz, że będziesz wieczny? Z~każdym kolejnym życiem tracisz jedno oczko w~łańcuszku, który wisi na twej szyi. Jak myślisz, jak szybko uda mi się go zacisnąć? Równie szybko co ty na szyi Aidy? Byłeś dobry w~duszeniu, prawda bracie?!

Dobył miecza tak szybko, że nie zdążyłem nawet zareagować. Nie ciął jednak by zabić, gdyby chciał mojej śmierci, ciąłby w~szyję. Upadłem na ziemię i~poczułem jak robi mi się zimno. Brunatna krew broczyła z~uciętej dłoni na liście dookoła mnie i~błyskała w~słońcu. Poczułem jak moje oczy gasną. Znowu pozwoliłem mu wygrać. Moja twarz wpadła wprost w~miękkie, złote liście.

\paraSep

Słyszę jej śpiew. Cichy, cienki niczym struna głos, przecinający pustkę ciemnego pokoju.

\begin{itquote}
mado kara mieru\\
sawayaka akikaze no\\
yama o~mawaru ya\\
ano kane no koe\\
yomei\\
ikubaku ka aru\\
koyoi hakanashi\\
inochi mijikashi\\
\end{itquote}

Jedna z~wielu melodii przeszłości. Echo pośrodku mej głowy, które rezonuje tak mocno, że nie pamiętam, kiedy został wydany pierwszy dźwięk. Wszystko się nakłada. Wszystkie życia jednocześnie. Macham w~kierunku przystani odpływając na statku. Na wojnę z~Persją, na wojnę z~Chinami, na wojnę z~Niemcami. Kolejne wojny, kolejna śmierć, kolejne zmarnowane życie. Kolejne „tym razem będzie inaczej”. I~być może ostatni jesienny wiatr na mej twarzy.

Znów słyszę walkę. Metal uderza o~metal, strzały świszczą. Tnie materiał i~mięso, kruszy kości. Przyszli. To dlatego mnie nie zabił. Chciał bym cierpiał. Tak jak on przeze mnie i~moją nierozwagę. Teraz będę widział śmierć tych wszystkich dobrych ludzi. Mojej żony i~syna. Moich przyjaciół, poddanych i~służby. Winnych temu, że urodzili się w~złym miejscu i~czasie. Ludzie wrzuceni żywcem w~tryby historii.

\paraSep

Obmywam twarz i~przecieram lustro z~pary. Jest tu piekielnie duszno. Era globalnego ocieplenia i~dwóch pór roku. Słyszę głos, wydobywający się z~ciemności.

---~Artur Gible? ---~widzę w~lustrze policjanta. ---~Pójdzie pan ze mną.

Albo mają swoje macki wszędzie i~już mnie znaleźli, albo mój nowy gospodarz nie był do końca kryształowy. Wymazywanie pamięci to w~dzisiejszych czasach popularna praktyka. Żaden wykrywacz kłamstw nie działa, kiedy przesłuchiwany niczego nie pamięta.

---~Mogę spytać w~jakim celu?

---~Idziesz czy nie? ---~widzę, jak powoli chwyta pistolet w~kaburze. Próbuje zgrywać rewolwerowca. Doskakuję do niego jednym susem i~wbijam mu klucz od samochodu w~tętnicę. Wydaje z~siebie ciche charknięcie. Raz, dwa, trzy. Trzy uderzenia serca, trzy coraz mniejsze fontanny jasnej krwi z~tętnicy. Upuszczam go na ziemię i~widzę faceta stojącego w~drzwiach łazienki z~oczami wytrzeszczonymi, jakby dostał deską w~potylicę.

---~O, kurwa… ---~duka cichym, przerażonym głosem.

---~O, kurwa ---~powtarzam patrząc za jego plecy, na dwóch policjantów zamawiających kawę.
