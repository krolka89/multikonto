\chapter{Życie XIV}
\chapterAuthor{Usmiech\_Niebios}

\begin{itquote}
Zgięcie. Wyprost.
\end{itquote}

Teoretycznie wiedziałem, że istnieje taka możliwość. Ba, być może nawet już przez to przeszedłem. Ciężko spamiętać.

\begin{itquote}
Zgięcie. Wyprost.
\end{itquote}

Czas jest relatywny. Im dłużej żyjesz, tym szybciej biegnie. W~wieku ośmiu lat, rok to jedna ósma życia, w~wieku osiemdziesięciu -- jedna osiemdziesiąta.

\begin{itquote}
Zgięcie. Wyprost.
\end{itquote}

Nie jestem pewien, jak wygląda to w~mojej sytuacji. Teoretycznie żyłem setki, a~nawet tysiące lat. Praktycznie -- to mój pierwszy tydzień życia. Lub bardziej precyzyjnie: pierwszy dzień w~pierwszym tygodniu życia.

\begin{itquote}
Zgięcie. Wyprost.
\end{itquote}

Jestem w~stanie zrozumieć rozmowy wokół siebie. Ledwo. Dialekt, którego używają, różni się od tego, którego kiedyś się uczyłem. Nie umiałbym go nazwać. Po jakimś czasie wszystko się zlewa.

\begin{itquote}
Zgięcie. Wyprost.
\end{itquote}

Ciała w~których się odradzałem posiadały pewne granice. Z~reguły szybko je testowałem. Pewność ręki, refleks, wysokość skoku, szybkość. Po raz pierwszy od bardzo dawna muszę tworzyć od początku materiał, którego będę używał.

\begin{itquote}
Zgięcie. Wyprost.
\end{itquote}

Moja dłoń zaczyna boleć. Jednak okres, w~którym mogę ćwiczyć jest krótki. Za krótki. Rodzice, doktorzy, pielęgniarki. Wszyscy mnie obserwują. Są tu również kamery, jednak jak dotąd, nic nie wspominano o~moim treningu. Muszę trenować rękę, żeby nauczyć się pisać, zanim zapomnę o~tym, co chcę sobie przekazać. 

\begin{itquote}
Zgięcie. Wyprost.
\end{itquote}

Nie mogę mówić. Nie mogę pisać. Setki wspomnień krążą w~mojej głowie. Gdybym mógł, spróbowałbym się zabić. Czekałoby mnie nowe życie, w~którym miałbym wpływ na to co się dzieje, albo wyzwolenie z~tego koszmaru. Obie opcje wydają się być kuszące.

\begin{itquote}
Zgięcie. Wyprost.
\end{itquote}

Nie przemyślałem jeszcze co zrobię w~tym życiu. Część umiejętności przechodzi. Umiem modulować głos, tak aby wyjść na swoje z~każdej sytuacji i~rozpoznawać emocje lepiej, niż najlepsi pokerzyści. Z~drugiej strony, po paru żywotach zupełnie zapomniałem, jak grać na skrzypcach. Zapomniałem o~wielu umiejętnościach, które kiedyś opanowałem.

\begin{itquote}
Zgięcie. Wyprost.
\end{itquote}

Część informacji jest spisana. Przeżyłem dostatecznie dużo, żeby wiedzieć jak bardzo jest to przydatne. Pozwala się choć trochę odciążyć, zapomnieć. Trzeba tylko pamiętać miejsca. Miejsca, miejsca, miejsca. Czy będę w~stanie znaleźć swoje sekrety, czy są już zniszczone i~rozkradzione?

\begin{itquote}
Ból. Przerwa.
\end{itquote}

Najgorsza jest jednostajność. Czuje się jakby mijały wieki. Kiedy będę mógł coś zrobić, jakoś zareagować, pokazać ile rzeczy potrafię? Za rok? Dwa? Czy w~ogóle opłaca mi się szybko ujawniać moje zdolności, czy moi rodzice zaprowadzą mnie do egzorcysty? Nie wiem, nie wiem, nie wiem.

\begin{itquote}
Wdech. Wydech.
\end{itquote}

A może na tym polega moja wyjątkowość. Może każdy ma zdolność do odradzania się, tylko wszyscy stają się dziećmi. Może każdy zapomina co potrafi. Co potrafił. Może w~ogóle istnieje tylko jedna dusza i~to ja -- ja! -- byłem swoim katem. Swoją ofiarą. Swoim przeznaczeniem. Gdybym miał więcej dowodów, więcej czasu, więcej wspomnień! 

\begin{itquote}
Wdech. Wydech.
\end{itquote}

Powinienem nauczyć się jak umierać. Powinienem umierać na zawołanie. Czemu tego nie zrobiłem? Czemu nie pamiętam jak to zrobić? 

\begin{itquote}
Wdech. Wydech.
\end{itquote}

Nauczyłem się za to obliczać ile czasu minęło. Precyzyjnie. Jakbym miał zegar z~tyłu głowy. W~czasie który wydaje mi się być wiecznością, minęły dwie minuty.

\begin{itquote}
Wdech. Wydech.
\end{itquote}

\begin{itquote}
Zgięcie. Wyprost.
\end{itquote}

Miliony możliwości. Liczba, która kurczy się z~każdą sekundą. Będąc architektem, zostaniesz architektem. Nie staniesz się piosenkarzem. Statystyka jest pewnikiem którego muszę się trzymać, inaczej wszystko się rozpadnie. 

\begin{itquote}
Zgięcie. Wyprost.
\end{itquote}

Statystycznie jestem geniuszem.

Testy IQ przygotowywane są dla określonych grup wiekowych. Oceniają iloraz na tle populacji. Mam co najmniej sto lat przewagi nad siedmiolatkami, jestem w~stanie znaleźć połączenia i~schematy minimum dwadzieścia razy szybciej. Prawdopodobnie jestem poza skalą, unoszę się nad dwustoma punktami, lepiej niż Einstein, lepiej niż Da~Vinci. To zresztą tylko puste słowa. Dźwięki. Symbole. 

Oni umarli. A~ja żyję.

Mogłem ich znać. Istnieje taka możliwość. Nie pamiętam. Być może to wina nieukształtowania mojego mózgu, zamiast komputera dostałem liczydło. Z~drugiej strony odczucia są tu o~wiele wyraźniejsze. Głód. Ból. Śmiech. Wielu ludzi zapłaciłoby miliony. by czuć tak mocno jak dziecko. Przynajmniej te pozytywne rzeczy. Umarłem kiedyś z~głodu. W~porównaniu z~głodem odczuwanym tutaj, to niewinna pieszczota.

\begin{itquote}
Zgięcie. Wyprost.
\end{itquote}

Uczucia się kończą. Nie zliczę, ile wielkich miłości przeżyłem. Po jakimś czasie każda następna staje się kopią, kalką, parodią. Spektaklem, który znam na pamięć. Wiem jakich słów użyć, aby rozkochać w~sobie każdą osobę. Wiem, których osób nie da się rozkochać, które instynktownie mnie nienawidzą.

Kiedyś próbowałem je zmienić. Dziś nie zależy mi na tym. Skończyło się kiedy zbezcześciłem szczątki jedynego człowieka, przez którego popełniłem samobójstwo. Wykopałem je z~grobu. Odlałem się. Spojrzałem na datę jego śmierci.
Czterdzieści lat. Po czterdziestu latach od jego śmierci, ktoś wykopał jego trumnę, wyciągnął zażółcone rozsypujące się kości i~odlał się na nie. Nie czując satysfakcji.

Poczułem obrzydzenie do siebie. Wyszedłem z~tego cmentarza i~nigdy więcej tam nie wróciłem. Pisali o~tym w~gazetach, ale nie czułem strachu. Zapomniałem o~nim.

Czasem zapominam, że zapomniałem. To dobre chwile. Czuje się żywy.

\begin{itquote}
Zgięcie. Wyprost.
\end{itquote}

Książki mieszają mi w~głowie. Zapominam, co było naprawdę, a~co jest tylko wytworem wyobraźni. Nie mogę czytać nic poza swoimi informacjami. 

\begin{itquote}
Zgięcie. Wyprost.
\end{itquote}

Kiedyś miałem schizofrenię. Słyszałem głos w~swojej głowie, który nie należał do mnie. Dwa wcielenia w~jednym ciele? Możliwe. Udało mi się go wyciszyć. Być może zabiłem go w~umyśle, być może popełniłem samobójstwo. Nie pamiętam. 

\begin{itquote}
Wdech. Wydech.
\end{itquote}

Ile zapomniałem? Czy powinienem zapominać? Nowe życie, nowa karta. Boje się że doprowadziłoby to tylko do powtarzania się cyklu. Wąż Uroboros. Symbol, który kiedyś wytatuowano mi na piersi. Czemu pamiętam takie detale, zamiast dat w~które się przenosiłem? Zamiast wiedzy, czy przenoszę się jedynie w~przyszłość, czy również w~przeszłość? Zapisałem to. Na pewno to gdzieś zapisałem.

Chciałbym pamiętać, gdzie to zapisałem.

\begin{itquote}
Wdech. Wydech.
\end{itquote}

Ile zależy tu ode mnie? Niektóre ciała są naturalnie bardziej szczęśliwe od innych, Może jest w~nich więcej jakiegoś hormonu? Mówię ciała, ponieważ dla mnie umysł jest wciąż ten sam. Ten sam, ten sam, ten sam.

Degeneracja.

To ciało jest chyba jednym z~najgorszych. Jest jak więzienie, jak skorupa, z~której nie mogę się wydostać. Być może jest na coś chore, być może skutki podróży po życiach zaczynają wreszcie do mnie docierać. Być może, być może, być może.

Nagle czuję, że miałem odnaleźć siebie! Moje poprzednie wcielenia kiedyś się spotkały! Podałem sobie nawet hasło, aby móc się znaleźć. Przynajmniej raz mi się nie udało.

Nie.

To niemożliwe. Niemożliwe. Niemożliwe. Mam wolną wolę. Nie jest ustalone co się wydarzy. Mogę zrobić wszystko. To wina tego ciała, jest jak klatka dla mojego umysłu. Znam rozwiązanie, widzę je, ale nie mogę do niego dojść na tym poziomie. Zapomniałem.

\begin{itquote}
Wdech. Wydech.
\end{itquote}

Czy jestem za mądry, czy zbyt głupi? Wydaje mi się że myślę tu jaśniej, ale jednocześnie czegoś tu brakuje, czegoś czego nie umiem opisać. Miałem to kiedyś, ten kształt, tę myśl, ten kolor. Nie można wymyślić nowego koloru, ale można odczuć brak jednego z~nich. 

Muszę nauczyć się autohipnozy, wytworzyć progi w~mózgu odblokowujące się przy określonych słowach. Stworzyć listę tych słów, posegregować wspomnienia i~umiejętności. Tłumaczyć sobie wszystko po kolei, każdy fragment poprzednich żyć osobno. Natłok informacji mnie przytłacza.

Im mądrzejszy człowiek, tym mniej ma spokoju. Głupcy i~ludzie prości podążają jedną ścieżką przez całe życie. Pewnie kiedyś też tak robiłem, ale w~miarę przezywania, wprowadzałem urozmaicenia. Ulepszenia. Nie umiem już wrócić do dawnej radości, a~przynajmniej nie w~innych ciałach. To ciało pozwala przeżywać to, o~czym miałem zapomnieć. Emocje.

Pięć minut.

Nie jestem w~stanie pojąć, ile miałbym czekać na możliwość wypowiedzenia pierwszego słowa, pierwszego kroku, pierwszego rysunku. To życie wydaje się być dłuższe niż wszystkie poprzednie.

\begin{itquote}
Wdech. Wydech.
\end{itquote}

Mógłbym w~nim tyle zdziałać. Rozumiem więcej niż we wszystkich poprzednich żywotach, które pamiętam. Ale za jaką cenę? 

Zbyt wysoką. 

Nie pamiętam już kiedy ostatni raz popełniałem samobójstwo. Być może nie chcę. Wiem, że ten żywot powinien -- nie, musi -- zostać zapomniany. To tylko pięć minut, pierdolone pięć minut, a~już tracę pamięć wyrazów, sens, pewniki. W~następnym życiu potrzebne mi jest tylko jedno wydarzenie, jedna mantra. Muszę ją odnaleźć.

Muszę.

\hspace{2em}\emph{Muszę.}

Muszę.

\hspace{2em}\emph{Wydech.}

Wydech.

\hspace{2em}\emph{Wydech.}
